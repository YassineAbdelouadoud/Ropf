% Generated by Sphinx.
\def\sphinxdocclass{report}
\documentclass[letterpaper,10pt,english]{sphinxmanual}
\usepackage[utf8]{inputenc}
\DeclareUnicodeCharacter{00A0}{\nobreakspace}
\usepackage{cmap}
\usepackage[T1]{fontenc}
\usepackage{babel}
\usepackage{times}
\usepackage[Sonny]{fncychap}
\usepackage{longtable}
\usepackage{sphinx}
\usepackage{multirow}



\title{ropf Documentation}
\date{January 14, 2015}
\release{}
\author{Yassine Abdelouadoud}
\newcommand{\sphinxlogo}{}
\renewcommand{\releasename}{Release}
\makeindex

\makeatletter
\def\PYG@reset{\let\PYG@it=\relax \let\PYG@bf=\relax%
    \let\PYG@ul=\relax \let\PYG@tc=\relax%
    \let\PYG@bc=\relax \let\PYG@ff=\relax}
\def\PYG@tok#1{\csname PYG@tok@#1\endcsname}
\def\PYG@toks#1+{\ifx\relax#1\empty\else%
    \PYG@tok{#1}\expandafter\PYG@toks\fi}
\def\PYG@do#1{\PYG@bc{\PYG@tc{\PYG@ul{%
    \PYG@it{\PYG@bf{\PYG@ff{#1}}}}}}}
\def\PYG#1#2{\PYG@reset\PYG@toks#1+\relax+\PYG@do{#2}}

\expandafter\def\csname PYG@tok@gd\endcsname{\def\PYG@tc##1{\textcolor[rgb]{0.63,0.00,0.00}{##1}}}
\expandafter\def\csname PYG@tok@gu\endcsname{\let\PYG@bf=\textbf\def\PYG@tc##1{\textcolor[rgb]{0.50,0.00,0.50}{##1}}}
\expandafter\def\csname PYG@tok@gt\endcsname{\def\PYG@tc##1{\textcolor[rgb]{0.00,0.27,0.87}{##1}}}
\expandafter\def\csname PYG@tok@gs\endcsname{\let\PYG@bf=\textbf}
\expandafter\def\csname PYG@tok@gr\endcsname{\def\PYG@tc##1{\textcolor[rgb]{1.00,0.00,0.00}{##1}}}
\expandafter\def\csname PYG@tok@cm\endcsname{\let\PYG@it=\textit\def\PYG@tc##1{\textcolor[rgb]{0.25,0.50,0.56}{##1}}}
\expandafter\def\csname PYG@tok@vg\endcsname{\def\PYG@tc##1{\textcolor[rgb]{0.73,0.38,0.84}{##1}}}
\expandafter\def\csname PYG@tok@m\endcsname{\def\PYG@tc##1{\textcolor[rgb]{0.13,0.50,0.31}{##1}}}
\expandafter\def\csname PYG@tok@mh\endcsname{\def\PYG@tc##1{\textcolor[rgb]{0.13,0.50,0.31}{##1}}}
\expandafter\def\csname PYG@tok@cs\endcsname{\def\PYG@tc##1{\textcolor[rgb]{0.25,0.50,0.56}{##1}}\def\PYG@bc##1{\setlength{\fboxsep}{0pt}\colorbox[rgb]{1.00,0.94,0.94}{\strut ##1}}}
\expandafter\def\csname PYG@tok@ge\endcsname{\let\PYG@it=\textit}
\expandafter\def\csname PYG@tok@vc\endcsname{\def\PYG@tc##1{\textcolor[rgb]{0.73,0.38,0.84}{##1}}}
\expandafter\def\csname PYG@tok@il\endcsname{\def\PYG@tc##1{\textcolor[rgb]{0.13,0.50,0.31}{##1}}}
\expandafter\def\csname PYG@tok@go\endcsname{\def\PYG@tc##1{\textcolor[rgb]{0.20,0.20,0.20}{##1}}}
\expandafter\def\csname PYG@tok@cp\endcsname{\def\PYG@tc##1{\textcolor[rgb]{0.00,0.44,0.13}{##1}}}
\expandafter\def\csname PYG@tok@gi\endcsname{\def\PYG@tc##1{\textcolor[rgb]{0.00,0.63,0.00}{##1}}}
\expandafter\def\csname PYG@tok@gh\endcsname{\let\PYG@bf=\textbf\def\PYG@tc##1{\textcolor[rgb]{0.00,0.00,0.50}{##1}}}
\expandafter\def\csname PYG@tok@ni\endcsname{\let\PYG@bf=\textbf\def\PYG@tc##1{\textcolor[rgb]{0.84,0.33,0.22}{##1}}}
\expandafter\def\csname PYG@tok@nl\endcsname{\let\PYG@bf=\textbf\def\PYG@tc##1{\textcolor[rgb]{0.00,0.13,0.44}{##1}}}
\expandafter\def\csname PYG@tok@nn\endcsname{\let\PYG@bf=\textbf\def\PYG@tc##1{\textcolor[rgb]{0.05,0.52,0.71}{##1}}}
\expandafter\def\csname PYG@tok@no\endcsname{\def\PYG@tc##1{\textcolor[rgb]{0.38,0.68,0.84}{##1}}}
\expandafter\def\csname PYG@tok@na\endcsname{\def\PYG@tc##1{\textcolor[rgb]{0.25,0.44,0.63}{##1}}}
\expandafter\def\csname PYG@tok@nb\endcsname{\def\PYG@tc##1{\textcolor[rgb]{0.00,0.44,0.13}{##1}}}
\expandafter\def\csname PYG@tok@nc\endcsname{\let\PYG@bf=\textbf\def\PYG@tc##1{\textcolor[rgb]{0.05,0.52,0.71}{##1}}}
\expandafter\def\csname PYG@tok@nd\endcsname{\let\PYG@bf=\textbf\def\PYG@tc##1{\textcolor[rgb]{0.33,0.33,0.33}{##1}}}
\expandafter\def\csname PYG@tok@ne\endcsname{\def\PYG@tc##1{\textcolor[rgb]{0.00,0.44,0.13}{##1}}}
\expandafter\def\csname PYG@tok@nf\endcsname{\def\PYG@tc##1{\textcolor[rgb]{0.02,0.16,0.49}{##1}}}
\expandafter\def\csname PYG@tok@si\endcsname{\let\PYG@it=\textit\def\PYG@tc##1{\textcolor[rgb]{0.44,0.63,0.82}{##1}}}
\expandafter\def\csname PYG@tok@s2\endcsname{\def\PYG@tc##1{\textcolor[rgb]{0.25,0.44,0.63}{##1}}}
\expandafter\def\csname PYG@tok@vi\endcsname{\def\PYG@tc##1{\textcolor[rgb]{0.73,0.38,0.84}{##1}}}
\expandafter\def\csname PYG@tok@nt\endcsname{\let\PYG@bf=\textbf\def\PYG@tc##1{\textcolor[rgb]{0.02,0.16,0.45}{##1}}}
\expandafter\def\csname PYG@tok@nv\endcsname{\def\PYG@tc##1{\textcolor[rgb]{0.73,0.38,0.84}{##1}}}
\expandafter\def\csname PYG@tok@s1\endcsname{\def\PYG@tc##1{\textcolor[rgb]{0.25,0.44,0.63}{##1}}}
\expandafter\def\csname PYG@tok@gp\endcsname{\let\PYG@bf=\textbf\def\PYG@tc##1{\textcolor[rgb]{0.78,0.36,0.04}{##1}}}
\expandafter\def\csname PYG@tok@sh\endcsname{\def\PYG@tc##1{\textcolor[rgb]{0.25,0.44,0.63}{##1}}}
\expandafter\def\csname PYG@tok@ow\endcsname{\let\PYG@bf=\textbf\def\PYG@tc##1{\textcolor[rgb]{0.00,0.44,0.13}{##1}}}
\expandafter\def\csname PYG@tok@sx\endcsname{\def\PYG@tc##1{\textcolor[rgb]{0.78,0.36,0.04}{##1}}}
\expandafter\def\csname PYG@tok@bp\endcsname{\def\PYG@tc##1{\textcolor[rgb]{0.00,0.44,0.13}{##1}}}
\expandafter\def\csname PYG@tok@c1\endcsname{\let\PYG@it=\textit\def\PYG@tc##1{\textcolor[rgb]{0.25,0.50,0.56}{##1}}}
\expandafter\def\csname PYG@tok@kc\endcsname{\let\PYG@bf=\textbf\def\PYG@tc##1{\textcolor[rgb]{0.00,0.44,0.13}{##1}}}
\expandafter\def\csname PYG@tok@c\endcsname{\let\PYG@it=\textit\def\PYG@tc##1{\textcolor[rgb]{0.25,0.50,0.56}{##1}}}
\expandafter\def\csname PYG@tok@mf\endcsname{\def\PYG@tc##1{\textcolor[rgb]{0.13,0.50,0.31}{##1}}}
\expandafter\def\csname PYG@tok@err\endcsname{\def\PYG@bc##1{\setlength{\fboxsep}{0pt}\fcolorbox[rgb]{1.00,0.00,0.00}{1,1,1}{\strut ##1}}}
\expandafter\def\csname PYG@tok@mb\endcsname{\def\PYG@tc##1{\textcolor[rgb]{0.13,0.50,0.31}{##1}}}
\expandafter\def\csname PYG@tok@ss\endcsname{\def\PYG@tc##1{\textcolor[rgb]{0.32,0.47,0.09}{##1}}}
\expandafter\def\csname PYG@tok@sr\endcsname{\def\PYG@tc##1{\textcolor[rgb]{0.14,0.33,0.53}{##1}}}
\expandafter\def\csname PYG@tok@mo\endcsname{\def\PYG@tc##1{\textcolor[rgb]{0.13,0.50,0.31}{##1}}}
\expandafter\def\csname PYG@tok@kd\endcsname{\let\PYG@bf=\textbf\def\PYG@tc##1{\textcolor[rgb]{0.00,0.44,0.13}{##1}}}
\expandafter\def\csname PYG@tok@mi\endcsname{\def\PYG@tc##1{\textcolor[rgb]{0.13,0.50,0.31}{##1}}}
\expandafter\def\csname PYG@tok@kn\endcsname{\let\PYG@bf=\textbf\def\PYG@tc##1{\textcolor[rgb]{0.00,0.44,0.13}{##1}}}
\expandafter\def\csname PYG@tok@o\endcsname{\def\PYG@tc##1{\textcolor[rgb]{0.40,0.40,0.40}{##1}}}
\expandafter\def\csname PYG@tok@kr\endcsname{\let\PYG@bf=\textbf\def\PYG@tc##1{\textcolor[rgb]{0.00,0.44,0.13}{##1}}}
\expandafter\def\csname PYG@tok@s\endcsname{\def\PYG@tc##1{\textcolor[rgb]{0.25,0.44,0.63}{##1}}}
\expandafter\def\csname PYG@tok@kp\endcsname{\def\PYG@tc##1{\textcolor[rgb]{0.00,0.44,0.13}{##1}}}
\expandafter\def\csname PYG@tok@w\endcsname{\def\PYG@tc##1{\textcolor[rgb]{0.73,0.73,0.73}{##1}}}
\expandafter\def\csname PYG@tok@kt\endcsname{\def\PYG@tc##1{\textcolor[rgb]{0.56,0.13,0.00}{##1}}}
\expandafter\def\csname PYG@tok@sc\endcsname{\def\PYG@tc##1{\textcolor[rgb]{0.25,0.44,0.63}{##1}}}
\expandafter\def\csname PYG@tok@sb\endcsname{\def\PYG@tc##1{\textcolor[rgb]{0.25,0.44,0.63}{##1}}}
\expandafter\def\csname PYG@tok@k\endcsname{\let\PYG@bf=\textbf\def\PYG@tc##1{\textcolor[rgb]{0.00,0.44,0.13}{##1}}}
\expandafter\def\csname PYG@tok@se\endcsname{\let\PYG@bf=\textbf\def\PYG@tc##1{\textcolor[rgb]{0.25,0.44,0.63}{##1}}}
\expandafter\def\csname PYG@tok@sd\endcsname{\let\PYG@it=\textit\def\PYG@tc##1{\textcolor[rgb]{0.25,0.44,0.63}{##1}}}

\def\PYGZbs{\char`\\}
\def\PYGZus{\char`\_}
\def\PYGZob{\char`\{}
\def\PYGZcb{\char`\}}
\def\PYGZca{\char`\^}
\def\PYGZam{\char`\&}
\def\PYGZlt{\char`\<}
\def\PYGZgt{\char`\>}
\def\PYGZsh{\char`\#}
\def\PYGZpc{\char`\%}
\def\PYGZdl{\char`\$}
\def\PYGZhy{\char`\-}
\def\PYGZsq{\char`\'}
\def\PYGZdq{\char`\"}
\def\PYGZti{\char`\~}
% for compatibility with earlier versions
\def\PYGZat{@}
\def\PYGZlb{[}
\def\PYGZrb{]}
\makeatother

\renewcommand\PYGZsq{\textquotesingle}

\begin{document}

\maketitle
\tableofcontents
\phantomsection\label{index::doc}


Contents:


\chapter{ropf package}
\label{ropf:ropf-package}\label{ropf:welcome-to-ropf-s-documentation}\label{ropf::doc}

\section{Submodules}
\label{ropf:submodules}

\section{ropf.Simulation module}
\label{ropf:ropf-simulation-module}\label{ropf:module-ropf.Simulation}\index{ropf.Simulation (module)}
These are the definitions of classes to obtain a complete simulation model
\index{DERModel (class in ropf.Simulation)}

\begin{fulllineitems}
\phantomsection\label{ropf:ropf.Simulation.DERModel}\pysiglinewithargsret{\strong{class }\code{ropf.Simulation.}\bfcode{DERModel}}{\emph{input\_folder}}{}
This class contains the description of the DERs in the network
\index{init\_dermodel() (ropf.Simulation.DERModel method)}

\begin{fulllineitems}
\phantomsection\label{ropf:ropf.Simulation.DERModel.init_dermodel}\pysiglinewithargsret{\bfcode{init\_dermodel}}{\emph{input\_folder}}{}
This method will populate a DERModel object attributes by obtaining the relevant
data in the input\_folder

\end{fulllineitems}

\index{pv\_inverter\_power (ropf.Simulation.DERModel attribute)}

\begin{fulllineitems}
\phantomsection\label{ropf:ropf.Simulation.DERModel.pv_inverter_power}\pysigline{\bfcode{pv\_inverter\_power}\strong{ = None}}
A vector of installed photovoltaic inverter powers

\end{fulllineitems}

\index{pv\_set\_points (ropf.Simulation.DERModel attribute)}

\begin{fulllineitems}
\phantomsection\label{ropf:ropf.Simulation.DERModel.pv_set_points}\pysigline{\bfcode{pv\_set\_points}\strong{ = None}}
An array of photovoltaic set points

\end{fulllineitems}

\index{storage\_inverter\_power (ropf.Simulation.DERModel attribute)}

\begin{fulllineitems}
\phantomsection\label{ropf:ropf.Simulation.DERModel.storage_inverter_power}\pysigline{\bfcode{storage\_inverter\_power}\strong{ = None}}
A vector of installed storage inverter powers

\end{fulllineitems}

\index{storage\_set\_points (ropf.Simulation.DERModel attribute)}

\begin{fulllineitems}
\phantomsection\label{ropf:ropf.Simulation.DERModel.storage_set_points}\pysigline{\bfcode{storage\_set\_points}\strong{ = None}}
A vector of storage set points

\end{fulllineitems}


\end{fulllineitems}

\index{LoadModel (class in ropf.Simulation)}

\begin{fulllineitems}
\phantomsection\label{ropf:ropf.Simulation.LoadModel}\pysiglinewithargsret{\strong{class }\code{ropf.Simulation.}\bfcode{LoadModel}}{\emph{input\_folder}}{}
This class contains the description of the loads in the network
\index{activeload (ropf.Simulation.LoadModel attribute)}

\begin{fulllineitems}
\phantomsection\label{ropf:ropf.Simulation.LoadModel.activeload}\pysigline{\bfcode{activeload}\strong{ = None}}
An array of active loads, with a line for each bus and a column for each time step

\end{fulllineitems}

\index{init\_loadmodel() (ropf.Simulation.LoadModel method)}

\begin{fulllineitems}
\phantomsection\label{ropf:ropf.Simulation.LoadModel.init_loadmodel}\pysiglinewithargsret{\bfcode{init\_loadmodel}}{\emph{input\_folder}}{}
This method will populate a LoadModel object attributes by obtaining the
relevant data in the input\_folder

\end{fulllineitems}

\index{reactiveload (ropf.Simulation.LoadModel attribute)}

\begin{fulllineitems}
\phantomsection\label{ropf:ropf.Simulation.LoadModel.reactiveload}\pysigline{\bfcode{reactiveload}\strong{ = None}}
An array of reactive loads,, with a line for each bus and a column for each time step

\end{fulllineitems}


\end{fulllineitems}

\index{Model (class in ropf.Simulation)}

\begin{fulllineitems}
\phantomsection\label{ropf:ropf.Simulation.Model}\pysiglinewithargsret{\strong{class }\code{ropf.Simulation.}\bfcode{Model}}{\emph{input\_folder}}{}
This class contains the data necessary for a simulation
\index{DERModel (ropf.Simulation.Model attribute)}

\begin{fulllineitems}
\phantomsection\label{ropf:ropf.Simulation.Model.DERModel}\pysigline{\bfcode{DERModel}\strong{ = None}}
A description of the DERs in the network

\end{fulllineitems}

\index{LoadModel (ropf.Simulation.Model attribute)}

\begin{fulllineitems}
\phantomsection\label{ropf:ropf.Simulation.Model.LoadModel}\pysigline{\bfcode{LoadModel}\strong{ = None}}
A description of the loads in the network

\end{fulllineitems}

\index{NetworkModel (ropf.Simulation.Model attribute)}

\begin{fulllineitems}
\phantomsection\label{ropf:ropf.Simulation.Model.NetworkModel}\pysigline{\bfcode{NetworkModel}\strong{ = None}}
A description of the network parameters

\end{fulllineitems}


\end{fulllineitems}

\index{NetworkModel (class in ropf.Simulation)}

\begin{fulllineitems}
\phantomsection\label{ropf:ropf.Simulation.NetworkModel}\pysiglinewithargsret{\strong{class }\code{ropf.Simulation.}\bfcode{NetworkModel}}{\emph{input\_folder}}{}
This class contains the data relative to the network
\index{current\_limit (ropf.Simulation.NetworkModel attribute)}

\begin{fulllineitems}
\phantomsection\label{ropf:ropf.Simulation.NetworkModel.current_limit}\pysigline{\bfcode{current\_limit}\strong{ = None}}
An array of line current limit

\end{fulllineitems}

\index{from\_node (ropf.Simulation.NetworkModel attribute)}

\begin{fulllineitems}
\phantomsection\label{ropf:ropf.Simulation.NetworkModel.from_node}\pysigline{\bfcode{from\_node}\strong{ = None}}
A vector of node IDs from which the lines are originating

\end{fulllineitems}

\index{init\_networkmodel() (ropf.Simulation.NetworkModel method)}

\begin{fulllineitems}
\phantomsection\label{ropf:ropf.Simulation.NetworkModel.init_networkmodel}\pysiglinewithargsret{\bfcode{init\_networkmodel}}{\emph{input\_folder}}{}
This method will populate a DERModel object attributes by obtaining the relevant
data in the input\_folder

\end{fulllineitems}

\index{oltc\_constraints (ropf.Simulation.NetworkModel attribute)}

\begin{fulllineitems}
\phantomsection\label{ropf:ropf.Simulation.NetworkModel.oltc_constraints}\pysigline{\bfcode{oltc\_constraints}\strong{ = None}}
A vector of 2 of or more elements. If its length is 2, a continuous model will be used for
the On-Load Tap Changer, with the two values representing up and down limit of voltage downstream of the OLTC.
If it is more than 2, a discrete model will be used, with the values representing the possible voltages
downstream of the OLTC

\end{fulllineitems}

\index{reactance (ropf.Simulation.NetworkModel attribute)}

\begin{fulllineitems}
\phantomsection\label{ropf:ropf.Simulation.NetworkModel.reactance}\pysigline{\bfcode{reactance}\strong{ = None}}
A vector of line reactance

\end{fulllineitems}

\index{resistance (ropf.Simulation.NetworkModel attribute)}

\begin{fulllineitems}
\phantomsection\label{ropf:ropf.Simulation.NetworkModel.resistance}\pysigline{\bfcode{resistance}\strong{ = None}}
A vector of line resistance

\end{fulllineitems}

\index{to\_node (ropf.Simulation.NetworkModel attribute)}

\begin{fulllineitems}
\phantomsection\label{ropf:ropf.Simulation.NetworkModel.to_node}\pysigline{\bfcode{to\_node}\strong{ = None}}
A vector of node IDs to which the lines are ending

\end{fulllineitems}

\index{voltage\_limit (ropf.Simulation.NetworkModel attribute)}

\begin{fulllineitems}
\phantomsection\label{ropf:ropf.Simulation.NetworkModel.voltage_limit}\pysigline{\bfcode{voltage\_limit}\strong{ = None}}
A vector of 2 elements representing the upper and lower bounds on voltage magnitude

\end{fulllineitems}


\end{fulllineitems}



\section{Module contents}
\label{ropf:module-contents}\label{ropf:module-ropf}\index{ropf (module)}

\chapter{Indices and tables}
\label{index:indices-and-tables}\begin{itemize}
\item {} 
\emph{genindex}

\item {} 
\emph{modindex}

\item {} 
\emph{search}

\end{itemize}


\renewcommand{\indexname}{Python Module Index}
\begin{theindex}
\def\bigletter#1{{\Large\sffamily#1}\nopagebreak\vspace{1mm}}
\bigletter{r}
\item {\texttt{ropf}}, \pageref{ropf:module-ropf}
\item {\texttt{ropf.Simulation}}, \pageref{ropf:module-ropf.Simulation}
\end{theindex}

\renewcommand{\indexname}{Index}
\printindex
\end{document}
